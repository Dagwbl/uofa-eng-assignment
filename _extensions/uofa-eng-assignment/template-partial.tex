% ============================================================================
% UofA Engineering Assignment LaTeX Header - Template Partial
% ============================================================================
% This file contains LaTeX commands to format University of Alberta 
% Engineering assignment documents. It's a "partial" template that can be
% included in other LaTeX documents.
%
% Based on the original uofa-eng-assignment.cls
% ============================================================================

% ----------------------------------------------------------------------------
% PAGE LAYOUT SETTINGS
% ----------------------------------------------------------------------------
% These commands control the dimensions of the text area and margins

% \setlength sets the value of a length variable in LaTeX
% \textwidth: width of the main text area (6.5 inches)
\setlength{\textwidth}{6.5in}

% \textheight: height of the main text area (9 inches)
\setlength{\textheight}{9.in}

% \oddsidemargin: left margin on odd-numbered pages (0 inches = default)
\setlength{\oddsidemargin}{0in}

% \headheight: height reserved for page headers (0 = no header space)
\setlength{\headheight}{0in}

% ----------------------------------------------------------------------------
% GEOMETRY PACKAGE - Advanced Page Layout
% ----------------------------------------------------------------------------
% The geometry package provides an easier way to set page margins
% [margin=2cm] sets all margins (top, bottom, left, right) to 2 centimeters
\usepackage[margin=2cm]{geometry}

% ----------------------------------------------------------------------------
% COLOR DEFINITIONS
% ----------------------------------------------------------------------------
% Define custom colors for syntax highlighting and other uses
% Colors are defined in RGB format with values from 0 to 1

\usepackage{color}

% Green color for comments in code (60% green, 0% red, 0% blue)
\definecolor{codegreen}{rgb}{0,0.6,0}

% Gray color for line numbers (50% of each color = medium gray)
\definecolor{codegray}{rgb}{0.5,0.5,0.5}

% Purple color for strings (58% red, 0% green, 82% blue)
\definecolor{codepurple}{rgb}{0.58,0,0.82}

% Light beige background color (95% red, 95% green, 92% blue)
\definecolor{backcolour}{rgb}{0.95,0.95,0.92}

% ----------------------------------------------------------------------------
% CODE LISTINGS CONFIGURATION (DISABLED BY DEFAULT)
% ----------------------------------------------------------------------------
% The listings package is commented out to avoid conflicts with Quarto's
% built-in syntax highlighting. Quarto uses its own system for code blocks.
%
% If you need to use raw LaTeX code blocks with \begin{lstlisting}, 
% uncomment the following lines. However, this may interfere with Quarto's
% markdown code blocks (```language).
%
% \usepackage{listings}
%
% \lstdefinestyle{mystyle}{
%     % Background color for code blocks
%     backgroundcolor=\color{backcolour},
%     % Color for comments in code
%     commentstyle=\color{codegreen},
%     % Color for keywords (like 'if', 'while', 'class')
%     keywordstyle=\color{magenta},
%     % Style for line numbers (tiny font, gray color)
%     numberstyle=\tiny\color{codegray},
%     % Color for strings in code
%     stringstyle=\color{codepurple},
%     % Base font size for code
%     basicstyle=\footnotesize,
%     % Don't break lines at whitespace only
%     breakatwhitespace=false,
%     % Allow automatic line breaking
%     breaklines=true,
%     % Position of caption (b = bottom)
%     captionpos=b,
%     % Preserve spaces in code
%     keepspaces=true,
%     % Show line numbers on the left
%     numbers=left,
%     % Distance between line numbers and code
%     numbersep=5pt,
%     % Don't show spaces as special characters
%     showspaces=false,
%     % Don't show spaces in strings as special characters
%     showstringspaces=false,
%     % Don't show tabs as special characters
%     showtabs=false,
%     % Tab size in spaces
%     tabsize=2
% }

% ----------------------------------------------------------------------------
% GRAPHICS AND IMAGE PACKAGES
% ----------------------------------------------------------------------------

% \usepackage loads external packages that add functionality to LaTeX

% graphicx: Enhanced support for including images
\usepackage{graphicx}

% fancybox: Creates fancy boxes around text (commented out)
% This package may conflict with verbatim environments (code blocks)
% \usepackage{fancybox}

% inputenc: Specifies input encoding (UTF-8 for Unicode characters)
% UTF-8 allows you to use special characters directly in your document
\usepackage[utf8]{inputenc}

% textgreek: Allows Greek letters in text mode (e.g., α, β, γ)
% Commented out by default, uncomment if you need Greek letters outside math mode
% \usepackage{textgreek}

% epsfig: Legacy package for including EPS graphics
% graphicx (loaded above) is more modern but epsfig is kept for compatibility
\usepackage{epsfig,graphicx}

% multicol: Allows text to be typeset in multiple columns
\usepackage{multicol}

% ----------------------------------------------------------------------------
% PSTRICKS PACKAGES (DISABLED BY DEFAULT)
% ----------------------------------------------------------------------------
% pstricks and pst-plot are powerful packages for creating graphics
% They are commented out because they may conflict with Quarto's syntax highlighting
% 
% Uncomment these if you need to create complex diagrams or plots in pure LaTeX,
% but be aware this may cause issues with Quarto code blocks
%
% \usepackage{multicol,pst-plot}
% \usepackage{pstricks}

% ----------------------------------------------------------------------------
% MATHEMATICS PACKAGES
% ----------------------------------------------------------------------------
% These packages provide enhanced mathematical typesetting capabilities

% amsmath: American Mathematical Society's math extensions
% Provides environments like align, gather, split for complex equations
\usepackage{amsmath}

% amsfonts: Additional mathematical fonts (e.g., blackboard bold: ℝ, ℂ, ℕ)
\usepackage{amsfonts}

% amssymb: Additional mathematical symbols (e.g., ∴, ∵, ⊂, ⊃)
\usepackage{amssymb}

% eucal: Euler calligraphic font (elegant script letters for math)
\usepackage{eucal}

% ----------------------------------------------------------------------------
% CUSTOM MATHEMATICAL OPERATORS AND COMMANDS
% ----------------------------------------------------------------------------
% Define custom commands to make writing mathematical expressions easier

% \DeclareMathOperator: Creates a new operator that appears in Roman font
% Tr: Trace operator (used in linear algebra and quantum mechanics)
\DeclareMathOperator{\tr}{Tr}

% \newcommand*: Defines a new command (the * means it doesn't allow paragraph breaks)
% \op{x}: Creates an operator with a check mark and bold font
% Example: \op{H} produces Ȟ in bold
\newcommand*{\op}[1]{\check{\mathbf#1}}

% Dirac notation commands for quantum mechanics:

% \bra{x}: Creates a "bra" vector ⟨x|
% Example: \bra{\psi} produces ⟨ψ|
\newcommand{\bra}[1]{\langle #1 |}

% \ket{x}: Creates a "ket" vector |x⟩
% Example: \ket{\psi} produces |ψ⟩
\newcommand{\ket}[1]{| #1 \rangle}

% \braket{x}{y}: Creates an inner product ⟨x|y⟩
% Example: \braket{\psi}{\phi} produces ⟨ψ|φ⟩
\newcommand{\braket}[2]{\langle #1 | #2 \rangle}

% \mean{x}: Creates an expectation value ⟨x⟩
% Example: \mean{E} produces ⟨E⟩ (mean energy)
\newcommand{\mean}[1]{\langle #1 \rangle}

% \opvec{x}: Creates an operator vector with check mark and vector arrow
% Example: \opvec{p} produces a momentum operator
\newcommand{\opvec}[1]{\check{\vec #1}}

% \renewcommand: Redefines an existing command
% \sp{x}: Creates a split equation environment
% This allows multi-line equations within display math mode
% Example: \sp{a &= b \\ &= c} produces aligned equations
\renewcommand{\sp}[1]{$${\begin{split}#1\end{split}}$$}

% ============================================================================
% CUSTOM TITLE PAGE FOR UOFA ENGINEERING ASSIGNMENTS
% ============================================================================
% This section creates a custom title page with student information and logo

% \makeatletter and \makeatother: Allow use of @ in command names
% In LaTeX, @ is normally not allowed in command names, but internal
% commands often use it. These commands temporarily change that rule.
\makeatletter

% ----------------------------------------------------------------------------
% DEFINE METADATA COMMANDS AND DEFAULT VALUES
% ----------------------------------------------------------------------------
% These commands define the student information that will appear on the title page

% \def: Defines a new command or redefines an existing one
% The following create warning messages if metadata is not provided:

% Student name - shows warning if not set
\def\@studentname{\@latex@warning@no@line{No \noexpand\studentname given}}

% Student ID - shows warning if not set
\def\@studentid{\@latex@warning@no@line{No \noexpand\studentid given}}

% Course name - shows warning if not set
\def\@course{\@latex@warning@no@line{No \noexpand\course given}}

% Assignment title - shows warning if not set
\def\@assignment{\@latex@warning@no@line{No \noexpand\assignment given}}

% Date - optional field, no warning if not provided
\def\@date{}

% Logo path - default to the built-in logo
% This can be overridden by providing a custom logo path in YAML
\def\@logopath{_extensions/uofa-eng-assignment/logo.png}

% ----------------------------------------------------------------------------
% DEFINE USER COMMANDS TO SET METADATA
% ----------------------------------------------------------------------------
% These commands allow users to set the metadata values

% \newcommand: Creates a new command
% [1]: Indicates the command takes 1 argument
% {#1}: Refers to the first (and only) argument

% Set student name
\newcommand{\studentname}[1]{\def\@studentname{#1}}

% Set student ID
\newcommand{\studentid}[1]{\def\@studentid{#1}}

% Set course name and code
\newcommand{\course}[1]{\def\@course{#1}}

% Set assignment number or title
\newcommand{\assignment}[1]{\def\@assignment{#1}}

% Set logo path (for custom logo support)
\newcommand{\logopath}[1]{\def\@logopath{#1}}

% \renewcommand: Redefine the existing \date command
\renewcommand{\date}[1]{\def\@date{#1}}

% ----------------------------------------------------------------------------
% REDEFINE \maketitle COMMAND
% ----------------------------------------------------------------------------
% \maketitle is the standard LaTeX command that creates the title page
% We redefine it here to create our custom layout

\renewcommand\maketitle{
    % -----------------------------------------------------------------------
    % LEFT SIDE: Student Information
    % -----------------------------------------------------------------------
    % \begin{flushleft}...\end{flushleft}: Align content to the left
    \begin{flushleft}
        % Display student name (\\: line break)
        Student Name: \@studentname \\
        % Display student ID
        Student ID: \@studentid
        % Conditional: Only show date if it was provided
        % \ifx tests if two values are identical
        % \@date\empty: true if date is empty
        % \else: if date is not empty, show it
        % \fi: end of conditional
        \ifx\@date\empty\else \\ Date: \@date \fi
    \end{flushleft}

    % -----------------------------------------------------------------------
    % RIGHT SIDE: University Logo
    % -----------------------------------------------------------------------
    % \begin{flushright}...\end{flushright}: Align content to the right
    % \vspace{-15mm}: Move up by 15 millimeters (negative = move up)
    % This pulls the logo up to align with the student info
    \begin{flushright}\vspace{-15mm}
        % \includegraphics: Insert an image
        % [height=2cm]: Scale image to 2 centimeters high (width scales proportionally)
        % \@logopath: Use the logo path (default or custom)
        \includegraphics[height=2cm]{\@logopath}
    \end{flushright}

    % -----------------------------------------------------------------------
    % CENTER: Course and Assignment Information
    % -----------------------------------------------------------------------
    % \begin{center}...\end{center}: Center the content
    % \vspace{-1cm}: Move up by 1 centimeter
    \begin{center}\vspace{-1cm}
        % \textbf: Bold text
        % \large: Larger font size
        % Display course name in bold and large font
        \textbf{\large \@course}\\
        % Display assignment number/title
        \@assignment
    \end{center}

    % -----------------------------------------------------------------------
    % HORIZONTAL RULE (Line Separator)
    % -----------------------------------------------------------------------
    % \rule: Draw a line
    % \linewidth: Make the line span the full width of the text
    % {0.1mm}: Line thickness (0.1 millimeter)
    \rule{\linewidth}{0.1mm}

    % -----------------------------------------------------------------------
    % SPACING AFTER TITLE
    % -----------------------------------------------------------------------
    % \bigskip: Add vertical space (approximately one blank line)
    % Two \bigskip commands = two blank lines
    \bigskip
    \bigskip
}

% \makeatother: Restore normal behavior (@ no longer allowed in command names)
\makeatother

% ============================================================================
% END OF TEMPLATE PARTIAL
% ============================================================================
% This file is now ready to be included in LaTeX documents.
% The \maketitle command will generate the custom title page when called.
% ============================================================================
